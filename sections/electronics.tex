The robot includes the following components:
\begin{itemize}
\item Intel NUC computer.
\item Arduino Mega 2560 + Motor Shield + custom I/O board.
\item PrimeSense RGB-D 1.09 sensor.
\item 2 motors + wheels
\item 4 Sharp GP2D120 (short range)
\item 2 Sharp GP2D12 (long range)
\item IMU Phidgets
\item Speaker + sound card
\item LiPo battery 3-cell 5000 mAh.
\end{itemize}

\subsection{IR sensors}
\subsubsection{Calibration}

Because of non-linear sensor output vs distance curve it is not practical to use raw values from IR sensors in higher level nodes. Therefore it was needed to create node which converts raw data from sensors to distance. As the raw output changes values more rapidly in lower distances, measurements in low range were made quite densely (by measuring sensor output every 0.5 cm) whereas in longer distances such accuracy were not needed because output values were approximately the same even within 5 cm range. Sensor converter node is the only one which subscribes to raw sensors node and other nodes in ROS subscribes directly to distance data.

\subsubsection{Filtering}

To filter out the noise from the IR sensors a set of independent Kalman Filters \cite{Thrun} is used due to its simplicity and ability to directly model the sensor and process noise. 
In particular, we consider the following process and measurement models:
\begin{align*}
x_{t+1} &= x_t + \epsilon \\
z_t &= x_t + \delta
\end{align*}
, where $\epsilon \sim \mathcal{N}(0,Q)$ and $\delta \sim \mathcal{N}(0,R)$. A value of $R = 0.01$, $Q = 0.04$ was chosen in order to effectively filter but not reduce the bandwidth (responsiveness) too much. 